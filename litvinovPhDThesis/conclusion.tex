\chapter*{Заключение}
\addcontentsline{toc}{chapter}{Заключение}
Основные результаты, полученные в данной диссертации, таковы.
\begin{enumerate}
	\item Разработана методика для создания предметно-ориентированных языков с помощью 
		графического языка метамоделирования и сопутствующих визуальных языков.
	\item Предложен новый способ метамоделирования: <<метамоделирование на лету>>, позволяющий
		создавать визуальный язык в процессе его использования.
	\item Предложенные методики реализованы в виде технологии на базе системы QReal.
	\item Проведена апробация разработанных методик и технологии при создании редактора, 
		генератора, средств проверки ограничений среды QReal:Robots и других предметно-ориентированных 
		решений.
\end{enumerate}

Исходя из особенностей предлагаемых методик можно дать следующие \textbf{рекомендации} 
по применению полученных результатов.
\begin{enumerate}
	\item Предложенные методики подходят для реализации небольших и средних по размерам 
		предметно-ориентированных языков.
	\item Наиболее эффективны предлагаемые методики в ситуации, когда нет чёткого представления
		о языке, который должен быть создан, но есть эксперт предметной области, участвующий
		в создании языка
\end{enumerate}

Перспективы дальнейшей работы таковы:.
\begin{itemize}
	\item реализация интеграции созданных инструментальных средств с существующими 
		средствами других коллективов;
	\item разработка средств задания семантики и правил генерации кода для режима 
		<<метамоделирования на лету>>;
	\item проведение экспериментов по количественной оценке эффективности предложенных 
		методик и средств в промышленных условиях и сравнению с существующими DSM-платформами.
\end{itemize}
