\chapter*{Заключение}
\addcontentsline{toc}{chapter}{Заключение}
Основные результаты, полученные в данной диссертации, таковы.
\begin{enumerate}
	\item Проведён анализ различных подходов к реализации визуальных языков и различных 
		технологических средств, их реализующих.
	\item Разработана методика и набор инструментальных средств для  создания предметно-ориентированных 
		языков с помощью графического языка метамоделирования и сопутствующих визуальных языков.
	\item Предложен новый способ метамоделирования: "`метамоделирование на лету"'.
	\item Предложенные методики и технологии реализованы в виде промышленного продукта QReal.
	\item Проведена апробация при создании редактора, генератора, средств проверки ограничений 
		среды QReal:Robots и других предметно-ориентированных решений.
\end{enumerate}

Реализация предложенных в данной диссертации методик осуществлялась коллективом студентов 
и аспирантов кафедры системного программирования СПбГУ в рамках проекта QReal. Хочется 
особо отметить вклад студентов, работавших над данным проектом под руководством автора 
диссертации: Абрамова Ивана Александровича, Дерипаска Анны Олеговны, Гудошниковой Анны Андреевны, 
Жуковой Беллы Юрьевны, Заболотных Елены Петровны, Занько Софьи Владимировны, Иванова Всеволода Юрьевича, 
Кузенковой Анастасии Сергеевны, Кузнецовой Марьи Юрьевны, Курбанова Рауфа Эльшад оглы, 
Назаренко Владимира Владимировича, Нефёдова Ефима Андреевича, Никольского Кирилла Андреевича, 
Осечкиной Марии Сергеевны, Птахиной Алины Ивановны, Пышновой Александры Витальевны, 
Савина Никиты Сергеевича, Соковиковой Натальи Алексеевны, Такун Евгении Игоревны, 
Тихоновой Марии Валерьевны, всех студентов, работавших под руководством Брыксина Тимофея Александровича, 
а также вклад Дмитрия Мордвинова и Ирины Брюхановой. Без них данная диссертация вряд ли 
была бы возможна.