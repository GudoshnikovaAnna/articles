\newcommand{\sfs}{\fontsize{10pt}{10pt}\selectfont}
\sfs % размер шрифта и расстояния между строками
\thispagestyle{empty}

\vspace{10mm}
\begin{flushright}
  \Large\textit{На правах рукописи}
\end{flushright}

\vspace{25mm}
\begin{center}
  \Large\bf{Литвинов Юрий Викторович}
\end{center}

\vspace{20mm}
\begin{center}
  {\bf \LARGE Методы и средства разработки графических предметно-ориентированных языков
\par}

\vspace{20mm}
{\Large
Специальность 05.13.11 ---\par
математическое и программное обеспечение\par
вычислительных машин, комплексов и компьютерных сетей
}

\vspace{15mm}
\LARGE Автореферат\par
\Large{диссертации на соискание учёной степени\par
кандидата технических наук}
\end{center}

\vspace{35mm}
\begin{center}
  \Large{Санкт-Петербург\par 2016}
\end{center}

\newpage
% оборотная сторона обложки
\thispagestyle{empty}

\small{
Работа выполнена на кафедре системного программирования федерального государственного
бюджетного образовательного учреждения высшего образования <<Санкт-Петербургский
государственный университет>>.
}

\begin{table} [h]  
  \begin{tabular}{ll}  
   \makecell[l]{\sfs  Научный руководитель: \vspace{2.20cm}} &
   \makecell*[{{p{11cm}}}]{\sfs
      \textbf{\sfs Терехов Андрей Николаевич} \\
      \sfs доктор физико-математических наук, профессор \\ 
      \sfs заведующий кафедрой системного программирования федерального государственного
      бюджетного образовательного учреждения высшего образования <<Санкт-Петербургский
      государственный университет>>.
   }

\vspace{3mm} \\

   \makecell[l]{\sfs Официальные оппоненты: \vspace{5.65cm}} &
   \makecell[{{p{11cm}}}]{
      \sfs \textbf{Штейнберг Борис Яковлевич,} \\
      \sfs доктор технических наук, старший научный сотрудник, \\
      \sfs профессор, заведующий кафедрой алгебры и дискретной математики, федеральное 
         государственное автономное образовательное учреждение высшего 
         образования <<Южный федеральный университет>>, \\ 
      \sfs \textbf{Котляров Всеволод Павлович,} \\
      \sfs кандидат технических наук, доцент, \\
      \sfs профессор кафедры <<Информационные и управляющие системы>>, федеральное 
         государственное автономное образовательное учреждение высшего образования 
         <<Санкт-Петербургский политехнический университет Петра Великого>>. \\
   }

\vspace{3mm} \\

   \makecell[l]{\sfs Ведущая организация:\vspace{0.90cm}} &
   \makecell*[{{p{11cm}}}]{\sfs Федеральное государственное бюджетное учреждение 
      науки Институт систем информатики им. А.П. Ершова Сибирского отделения Российской академии наук.
   }
  \end{tabular}  
\end{table}

\small{
\noindent Защита состоится 14 апреля 2016~г.~в~15 часов 30 минут на~заседании диссертационного 
совета Д212.232.51 на базе Санкт-Петербургского государственного университета по 
адресу: 198504, Санкт-Петербург, Старый Петергоф, Университетский пр., 28, математико-механический 
факультет, ауд. 405.
}

\vspace{5mm}

\small{
\noindent С диссертацией можно ознакомиться в Научной библиотеке им. М. Горького Санкт-Петербургского 
государственного университета по адресу: 199034, Санкт-Петербург, Университетская  наб., 
7/9  и на сайте http://spbu.ru/science/disser/.
}

\vspace{5mm}
\noindent{Автореферат разослан <<\underline{\;\;\;\;\;\;\;}>> 
\underline{\;\;\;\;\;\;\;\;\;\;\;\;\;\;\;\;\;\;\;\;} 20\underline{\;\;\;\;\;\;\;} года.}

\vspace{5mm}
\begin{table} [h]
  \begin{tabular}{p{8cm}cr}
    \begin{tabular}{p{8cm}}
      \sfs \small{Ученый секретарь}  \\
      \sfs диссертационного совета  \\
      \sfs Д212.232.51, д.ф.-м.н., проф.
    \end{tabular}
    & \begin{tabular}{c}
       %\includegraphics [height=2cm] {signature} 
    \end{tabular} 
    & \begin{tabular}{r}
       \\
       \\
       \sfs Демьянович Юрий Казимирович
    \end{tabular} 
  \end{tabular}
\end{table}

\newpage