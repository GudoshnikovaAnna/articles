\newcommand{\sfs}{\fontsize{12pt}{12pt}\selectfont}
\sfs % размер шрифта и расстояния между строками
\thispagestyle{empty}

\vspace{10mm}
\begin{flushright}
  \Large\textit{На правах рукописи}
\end{flushright}

\vspace{25mm}
\begin{center}
{\Large\bfЛитвинов Юрий Викторович}
\end{center}

\vspace{20mm}
\begin{center}
{\bf \LARGE Средства разработки визуальных предметно-ориентированных языков
\par}

\vspace{20mm}
{\Large
Специальность 05.13.11 ---\par
математическое и программное обеспечение\par
вычислительных машин, комплексов, систем и сетей
}

\vspace{15mm}
\LARGEАвтореферат\par
\Largeдиссертации на соискание учёной степени\par
кандидата технических наук
\end{center}

\vspace{35mm}
\begin{center}
{\LargeСанкт-Петербург\par 2015}
\end{center}

\newpage
% оборотная сторона обложки
\thispagestyle{empty}

\small{
Работа выполнена на кафедре системного программирования федерального государственного
бюджетного образовательного учреждения высшего образования "Санкт-Петербургский
государственный университет".
}

\begin{table} [h]  
  \begin{tabular}{ll}  
   \makecell[l]{\sfs  Научный руководитель:\\~} &
   \makecell*[{{p{11cm}}}]{\sfs
      \textbf{\sfs Терехов Андрей Николаевич} \\
      доктор физико-математических наук, профессор \\ 
      заведующий кафедрой системного программирования федерального государственного
      бюджетного образовательного учреждения высшего образования "Санкт-Петербургский
      государственный университет".
   }

\vspace{3mm} \\

   \makecell[l]{\sfs Официальные оппоненты: \vspace{8.10cm}} &
   \makecell[{{p{11cm}}}]{
   \sfs \textbf{Штейнберг Борис Яковлевич,} \\
   \sfs доктор технических наук, старший научный сотрудник, \\
   \sfs заведующий кафедрой алгебры и дискретной математики, федеральное 
      государственное автономное образовательное учреждение высшего 
      образования <<Южный федеральный университет>>, \\ 


   \vspace{1mm} \\
   \sfs \textbf{Котляров Всеволод Павлович,} \\
   \sfs кандидат технических наук, доцент, \\
   \sfs профессор кафедры <<Информационные и управляющие системы>>, федеральное 
      государственное автономное образовательное учреждение высшего образования 
      <<Санкт-Петербургский политехнический университет Петра Великого>>, \\
   }

\vspace{3mm} \\

   \makecell[l]{\sfs Ведущая организация:\vspace{1.20cm}} &
   \makecell*[{{p{11cm}}}]{\sfs Федеральное государственное бюджетное учреждение 
      науки Институт систем информатики им. А.П. Ершова Сибирского отделения Российской академии наук
   }
  \end{tabular}  
\end{table}

\small{
\noindent Защита состоится DD mmmmmmmm YYYY~г.~в~XX часов на~заседании диссертационного 
совета Д212.232.51 на базе Санкт-Петербургского государственного университета по 
адресу: 198504, Санкт-Петербург, Старый Петергоф, Университетский пр., 28, математико-механический 
факультет, ауд. 405.
}

\vspace{5mm}

\small{
\noindent С диссертацией можно ознакомиться в Научной библиотеке им. М. Горького Санкт-Петербургского 
государственного университета по адресу: 199034, Санкт-Петербург, Университетская  наб., 
7/9  и на сайте http://spbu.ru/disser2/disser/.........Dissert.pdf.
}

\vspace{5mm}
\noindentАвтореферат разослан DD mmmmmmmm YYYY года.

\vspace{5mm}
\begin{table} [h]
  \begin{tabular}{p{8cm}cr}
    \begin{tabular}{p{8cm}}
      \sfs \small{Ученый секретарь}  \\
      \sfs диссертационного совета  \\
      \sfs Д212.232.51, д.ф.-м.н.
    \end{tabular}
    & \begin{tabular}{c}
       %\includegraphics [height=2cm] {signature} 
    \end{tabular} 
    & \begin{tabular}{r}
       \\
       \\
       \sfs Демьянович Юрий Казимирович
    \end{tabular} 
  \end{tabular}
\end{table}

\newpage