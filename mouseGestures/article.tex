\documentclass[a5paper]{article}
\usepackage[a5paper, top=17mm, bottom=17mm, left=17mm, right=17mm]{geometry}
\usepackage[utf8]{inputenc}
\usepackage[T2A,T1]{fontenc}
\usepackage[colorlinks,filecolor=blue,citecolor=green,unicode,pdftex]{hyperref}
\usepackage{cmap}
\usepackage[english,russian]{babel}
\usepackage{amsmath}
\usepackage{amssymb,amsfonts,textcomp}
\usepackage{color}
\usepackage{array}
\usepackage{hhline}
\hypersetup{colorlinks=true, linkcolor=blue, citecolor=blue, filecolor=blue, urlcolor=blue, pdftitle=1, pdfauthor=, pdfsubject=, pdfkeywords=}
% \usepackage[pdftex]{graphicx}
\usepackage{graphicx}
\usepackage{epigraph}
% Раскомментировать тем, у кого этот пакет есть. Шрифт станет заметно красивее.
%\usepackage{literat}
\usepackage{indentfirst}

\sloppy
\pagestyle{plain}
%\pagestyle{empty}

\title{Мышиные жесты, ПЫЩЬ}

\author{Т.А. Брыксин \and Ю.В. Литвинов \and М.С. Осечкина}
\date{}
\begin{document}

\maketitle
\thispagestyle{empty}

\epigraph{Цивилизация движется вперед путем увеличения числа операций, которые мы можем осуществлять, не раздумывая над ними}%
         {Альфред Норт Уайтхед}

\begin{quote}
\small\noindent
Аннотация, чо
\end{quote}

\section*{Введение}
Одной из особенностей разработки, управляемой моделями (model-driven development, MDD), является активное использование 
визуальных языков. Практически все действия, выполняемые в CASE-средствах или других используемых инструментах так или 
иначе сводятся к манипуляциям над элементами этих языков и связями между ними. 

Эффективность любого используемого инструмента определяется тем, насколько удобно и быстро он позволяет выполнять те операции, 
для которых этот инструмент предназначен. В процессе разработки моделей одними из наиболее часто выполняемых действий над объектами 
на диаграммах являются их создание и удаление.  В большинстве CASE-средств для того, чтобы создать нужный объект на диаграмме, 
необходимо найти его либо на панели инструментов, либо выбрать в меню, а затем указать место на диаграмме, где бы мы хотели этот 
элемент разместить. В большинстве инструментариев также возможен вариант создания объектов «перетаскиванием» (drag and drop) их из 
палитры элементов соответствующей диаграммы. То есть даже для такой базовой операции как создание нового элемента разработчику нужно 
совершить не только набор чисто механических действий, но еще и, скажем, вспомнить, на какой вкладке палитры или в каком меню находится 
нужный ему элемент, тем самым переключая контекст с продумывания особенности иерархии создаваемых моделей на особенности манипуляции 
используемым инструментом. Нам кажется, что данную операцию можно и нужно автоматизировать, причем сделать это максимально удобным для 
пользователей CASE-средств. В данной статье в качестве такого решения рассматривается использование жестов мышью.

\section{Постановка задачи}
В качестве CASE-пакета, в котором было решено попытаться внедрить данный подход, была выбрана среда QReal, разрабатываемая на кафедре 
системного программирования СПбГУ. 

\end{document}
