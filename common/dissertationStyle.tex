%%% Макет страницы %%%
\geometry{a4paper,top=2cm,bottom=2cm,left=25mm,right=1cm}

%%% Кодировки и шрифты %%%
%\renewcommand{\rmdefault}{ftm} % Включаем Times New Roman

%%% Выравнивание и переносы %%%
\sloppy					% Избавляемся от переполнений
\clubpenalty=10000		% Запрещаем разрыв страницы после первой строки абзаца
\widowpenalty=10000		% Запрещаем разрыв страницы после последней строки абзаца
\hyphenpenalty=2000

%%% Библиография %%%
\makeatletter
\bibliographystyle{ugost2003s}	% Оформляем библиографию в соответствии с ГОСТ 7.1-2003
\def\bbl@main@language{russian}
\renewcommand{\@biblabel}[1]{#1.}	% Заменяем библиографию с квадратных скобок на точку:
\makeatother

% Объявляем дополнительную библиографию для ВАКовских статей и статей из Scopus, надо в автореферате
% dummy нужен, потому что в \newcites первая библиография должна быть обязательно негруппированной, just because
\newcites{dummy,vak,scopus}%
  {Placeholder,%
  {\large В изданиях из перечня российских рецензируемых научных %
  журналов, в которых должны быть опубликованы основные научные результаты %
  диссертаций на соискание ученых степеней доктора и кандидата наук},%
  {\large В изданиях, индексируемых в реферативных базах Scopus и Web Of Science}}

% Определяем для них всех стиль
\bibliographystylevak{ugost2003ssynopsis}
\bibliographystylescopus{ugost2003ssynopsis}

%%% Изображения %%%
\graphicspath{{images/}} % Пути к изображениям

%%% Цвета гиперссылок %%%
\definecolor{linkcolor}{rgb}{0.4,0,0}
\definecolor{citecolor}{rgb}{0,0.6,0}
\definecolor{urlcolor}{rgb}{0,0,1}
\hypersetup{
    colorlinks, linkcolor={linkcolor},
    citecolor={citecolor}, urlcolor={urlcolor}
}

%%% Оглавление %%%
%\renewcommand{\cftchapdotsep}{\cftdotsep}

% Таблицы покрасивее
\tabulinesep=1.2mm

% Печать через полтора интервала (ГОСТ Р 7.0.11-2011, 5.3.6)
\linespread{1.3}

% Номера страниц по центру сверху (ГОСТ Р 7.0.11-2011, 5.3.8)
\setlength{\headheight}{15pt}

\fancyhf{}
\fancyhead[C]{\thepage}
\pagestyle{fancy}
\renewcommand{\headrulewidth}{0pt}

\fancypagestyle{plain}{
\fancyhf{}
\fancyhead[C]{\thepage}
}

\AtBeginEnvironment{acronym}{%
  \renewcommand*\descriptionlabel[1]{\hspace\labelsep
                                \normalfont\bfseries\color{blue} #1}}

\titleformat{\chapter}[display]
  {\normalfont\large\bfseries\filcenter}{\chaptertitlename\ \thechapter}{20pt}{\Large}