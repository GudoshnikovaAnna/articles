\documentclass[a5paper]{article}
\usepackage[a5paper, top=17mm, bottom=17mm, left=17mm, right=17mm]{geometry}
\usepackage[utf8]{inputenc}
\usepackage[T2A,T1]{fontenc}
\usepackage[colorlinks,filecolor=blue,citecolor=green,unicode,pdftex]{hyperref}
\usepackage{cmap}
\usepackage[english,russian]{babel}
\usepackage{amsmath}
\usepackage{amssymb,amsfonts,textcomp}
\usepackage{color}
\usepackage{array}
\usepackage{hhline}
\hypersetup{colorlinks=true, linkcolor=blue, citecolor=blue, filecolor=blue, urlcolor=blue, pdftitle=1, pdfauthor=, pdfsubject=, pdfkeywords=}
% \usepackage[pdftex]{graphicx}
\usepackage{graphicx}
% \usepackage{epigraph}
% Раскомментировать тем, у кого этот пакет есть. Шрифт станет заметно красивее.
%\usepackage{literat}
\usepackage{indentfirst}
\usepackage{multirow}
\usepackage{subfig}

\sloppy
\pagestyle{plain}
%\pagestyle{empty}

\title{Технология визуального предметно-ориентированного проектирования и разработки ПО QReal}

\author{А.Я.Кириленко \and Наталья Вальтман}
\date{}
\begin{document}

\maketitle
\thispagestyle{empty}

\begin{quote}
\small\noindent
абстракт
\end{quote}

\section*{Введение}

Модельно-ориентированная разработка ПО основывается на представлении программы в виде набора моделей, представляющих ее с различных точек зрения. При этом обычно используются визуальные языки моделирования, с их помощью создаются разного уровня абстракции описания предметной области, разрабатываемой системы и взаимодействующего с ней окружения. Считается, что в целом данный подход упрощает процесс разработки и понимания системы, делает его более наглядным. Наиболее активное распространение CASE-технологий [сноска] началось в середине 90-х годов прошлого века, когда появился унифицированный язык моделирования UML [ссылка]. К концу 90-х годов был разработан набор методологий разработки ПО (и поддерживающих их инструментариев), в том или ином виде предполагающих активное использование визуального проектирования. Среди них [примеры]. Однако, желание иметь полную автоматическую генерацию исполняемого кода по диаграммам неизбежно влечет к жесткой формализации соответствующих графических языков. При этом использование языков общего назначения чаще всего приводит к тому, что диаграммы теряют наглядность и простоту. Парадигма предметно-ориентированного моделирования же основывается на том факте, что чаще создание нового специального языка и решение с его помощью поставленной практической задачи можно осуществить быстрее, чем решать ту же задачу с помощью языков общего назначения. Имея соответствующую инструментальную поддержку, данный подход позволяет значительно повысить уровень абстракции, с которым работают проектировщики, и увеличить производительность их труда в несколько раз ([ссылки на статьи из Книжки]).

\section{Предыстория}

Разработкой технологии QReal занимается научно-исследовательская группа изучения технологий визуального моделирования кафедры системного программирования Санкт-Петербургского Государственного Университета. Проект QReal базируется на результатах, полученных на кафедре и в лаборатории системного программирования под руководством проф. А.Н. Терехова с 1984 года. Первые разработанные графические редакторы создавались для поддержки языка SDL (рекомендация МККТТ Z.100 [ссылки]), к концу 80-х годов была реализована генерация программ управления телефонными станциями, генерация баз данных и сложных форм ввода/вывода к ним — технология RTST [ссылки]. 

К середине 90-х годов группой была реализована поддержка появившегося в то время языка UML 1.4 в виде технологии REAL [ссылки]. Технология не создавалась как универсальная, а была ориентирована на создание информационных систем, ориентированных на данные (диаграммы классов, объектов, задание ограничений с помощью OCL), а также создания встроенных систем реального времени (диаграммы классов, последовательностей, конечных автоматов). 

Технология QReal [ссылки] изначально задумывалась как развитие технологии REAL, основывающееся на использовании более современной версии языка UML --- 2.0. При этом на разрабатываемые средства накладывались требования многоплатформенности (возможность работы на наиболее популярных операционных системах MS Windows и Linux), поддержка многопользовательской разработки, возможность удаленного доступа к репозиторию системы и другая актуальная для сред визуального проектирования ПО функциональность.

Тут еще немного про DSM вижн.

\section{QReal}




\end{document}