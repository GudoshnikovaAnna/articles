\documentclass[a5paper]{article}
\usepackage[a5paper, top=17mm, bottom=17mm, left=17mm, right=17mm]{geometry}
\usepackage[utf8]{inputenc}
\usepackage[T2A,T1]{fontenc}
\usepackage[colorlinks,filecolor=blue,citecolor=green,unicode,pdftex]{hyperref}
\usepackage{cmap}
\usepackage[english,russian]{babel}
\usepackage{amsmath}
\usepackage{amssymb,amsfonts,textcomp}
\usepackage{color}
\usepackage{array}
\usepackage{hhline}
\hypersetup{colorlinks=true, linkcolor=blue, citecolor=blue, filecolor=blue, urlcolor=blue, pdftitle=1, pdfauthor=, pdfsubject=, pdfkeywords=}
% \usepackage[pdftex]{graphicx}
\usepackage{graphicx}
%\usepackage{epigraph}
% Раскомментировать тем, у кого этот пакет есть. Шрифт станет заметно красивее.
\usepackage{literat}
\usepackage{indentfirst}

\sloppy
\pagestyle{plain}
%\pagestyle{empty}

\title{Об управлении студенческими проектами}

\author{Т.А. Брыксин}
\date{}
\begin{document}

\maketitle
\thispagestyle{empty}

%\epigraph{покайтесь! ибо конец уже близок!}%
%         {Тема xmpp-конференции проекта QReal}

\begin{quote}
\small\noindent
Студенческий проект --- это некоммерческий проект по разработке программного обеспечения, в котором студенты могут под руководством опытных специалистов получить опыт в условиях, максимально приближенных к реальным промышленным или исследовательским проектам. В данной статье рассматриваются особенности организации и управления студенческими проектами с точки зрения непосредственных руководителей, приводится описание и статистика по нескольким таким проектам, сравниваются  используемые в различных проектах практики, делаются выводы об эффективности применения этих практик в зависимости от целей проекта.
\end{quote}

\section*{Введение}
Программная инженерия --- сравнительно молодая динамично развивающаяся область, поэтому к моменту завершения студентом обучения довольно большая часть полученных практических знаний успевает потерять актуальность. Кроме того, довольно многие знания и навыки, необходимые специалистам, не могут быть получены студентами из университетских курсов. Проводимые практические занятия, курсовые и дипломные работы не дают студенту достаточно практики реального  промышленного программирования, чтобы по завершении обучения быть полноценным специалистом в области программной инженерии. Одной из активностей, направленной на повышение уровня подготавливаемых специалистов, проводящейся кафедрой системного программирования СПбГУ совместно с ЗАО "Ланит-Терком", является практика проведения студенческий проектов и летних школ.

Студенческий проект представляет собой некоммерческий проект (как правило, с открытым исходным кодом), в котором группы студентов (порядка 4 - 12 человек) решают какую-либо реальную задачу. Студентами руководят, как правило, два специалиста компании, длительность студенческого проекта --- чуть менее семестра. Рабочие места студентам не предоставляются, так что большая часть работы выполняется студентами дома. Как правило, проводятся регулярные встречи участников проекта.

Компания рассматривает студенческие проекты в основном как средство отбора и подготовки кадров (подробнее об экономических аспектах подобной организации поиска кадров см. в [наверняка существующей тереховской публикации по этому вопросу, по крайней мере, доклад на сифуде Он делал]). Кафедра использует студенческие проекты также как средство отбора талантливых студентов, и для повышения уровня подготовки студентов уже отобранных. Мотивация студентов и руководителей проектов бывает различной, но в любом случае успешно прошедшие студпроект студенты могут получить рекомендацию для поступления на кафедру и на работу. Подробнее об организации студенческих проектов кафедрой системного программирования СПбГУ см. в~\cite{gagarsky}.

В данной статье представлены описания некоторых студенческих проектов, а также рассмотрены практики, в них применявшиеся. Представлены выводы, сделанные руководителями после сравнения нескольких проектов с разными поставленными задачами и разным набором применявшихся практик, а также объективные данные, такие, как число студентов, успешно окончивших студпроект, число написанных студентами за время участия в проекте строк кода и т.д. Изложенные в этой статье выводы могут быть интересны и для сообщества разработчиков свободного программного обеспечения --- студенческие проекты имеют много общего с ``обычными'' open source-проектами. Следует, однако, помнить, что есть и существенные различия --- во-первых, несмотря на то, что студенты работают в основном дома (что позволяет говорить о некоей распределённости команды), всё же довольно часты личные встречи, кроме того, не возникает проблем с языковыми и культурными различиями, и с часовыми поясами. Во-вторых, студенческий проект зачастую оказывается более организованной деятельностью, с чёткими сроками, задачами, руководителями. В-третьих, мотивация участников open source-проектов и студенческих проектов может сильно различаться. Всё это не позволяет напрямую переносить полученный опыт на более общий случай open source, но тем не менее, мы считаем, что часть сделанных выводов будет полезна.

\section{Критерии сравнения проектов}
Дальнейшее изложение построено в виде сравнения студенческих проектов по ряду признаков:
\begin{description}
	\item[Задачи] --- какие задачи ставили перед собой руководители студпроекта и какие цели преследовали. Как правило, руководители ставят перед собой сразу несколько задач. Цели студенческих проектов могут существенно различаться, от обучения студентов до подготовки кадров в конкретный коммерческий проект.
	\item[Технологии] --- технические средства, используемые при разработке. В этой секции будут перечислены используемые в проекте языки программирования, платформа, компиляторы, среды разработки и т.д. Эта информация, в частности, поможет оценить трудозатраты и объём реализованной функциональности по приведённым данным по числу написанных строк кода.
	\item[Организация процесса управления] --- здесь описываются методики распределения задач и контроля за ходом проекта, также вкратце описывается принцип построения команды, приводятся наблюдения, какими достоинствами или недостатками обладает выбранный в проекте принцип организации.
	\item[Коммуникации] --- особенности организации коммуникаций в проекте. Зачастую в начале проекта участники не знают друг друга, сильно отличаются по уровню подготовки (в студпроекте участвуют студенты с 1-го по 4-й курсы), обладают или не обладают опытом работы в команде и т.д., поэтому от эффективности установившихся практик обмена информацией может зависеть довольно многое. Это может быть особенно важно, если студенческий проект имеет целью создание или расширения сообщества, занимающегося разработкой какого-либо программного продукта. К тому же, при обмене опытом между руководителями проектов выяснилось, что подходы к организации коммуникаций могут весьма сильно разниться.
	\item[Ревью, модерация коммитов] --- особенности организации обратной связи. Также весьма важный с точки зрения психологии аспект организации работы.
	\item[Документирование] --- практики передачи и создания отчуждаемых знаний в проекте. Насколько активно они использовались и какое влияние по мнению руководителей оказали на результат.
	\item[Метрики] --- количественные характеристики, описывающие результаты проекта. Сколько всего студентов пришло в студпроект, сколько ушло, не успев погрузиться в проект, сколько ушло, после того, как приступило к работе, сколько успешно закончило, сколько строк кода было написано студентами за время проекта, сколько из них относилось к реализуемой в проекте задаче, а не к учебным упражнениям, сколько всего независимых задач было решено (или реализовано единиц функциональности), наконец, потраченное руководителями время.
\end{description}

\section{Описания проектов}
\subsection{QReal}
\subsubsection{Краткое описание проекта}
QReal~\cite{qreal} --- это проект по разработке системы визуального моделирования. Разрабатываемая система представляет собой набор редакторов диаграмм и генераторов исходного кода, объединённых общей инфраструктурой. Отличием QReal от аналогичных CASE-систем является возможность быстрого добавления поддержки новых визуальных языков описанием их синтаксиса на специализированном декларативном языке и генерации плагинов-редакторов по этому описанию. Работа над проектом ведётся с 2007 года и базируется на опыте предыдущих разработок, проводимых на кафедре под руководством проф. А.Н. Терехова. С 2008 года организуются студенческие проекты, направленные в основном на расширение функциональности графического ядра системы. Таким образом, студенческие проекты являются частью работы над уже существующим и развивающимся продуктом. Помимо студпроектов, в рамках работы над QReal пишется довольно большое количество курсовых и дипломных работ, идёт работа над кандидатскими диссертациями. Однако, несмотря на исследовательскую направленность проекта, его организация максимально приближена к промышленному процессу, типичному для Ланит-Терком. Руководили проектом два специалиста Ланит-Терком.

\subsubsection{Задачи}
Студпроекты, проводимые в рамках деятельности исследовательской группы QReal, имеют следующие цели, расположенные в порядке убывания приоритета:
\begin{itemize}
	\item Обучение студентов --- в студпроекте студенты получают опыт работы в условиях, максимально приближенных к индустриальным, а также опыт ведения исследовательской деятельности.
	\item Подготовка новых членов исследовательской группы --- в процессе работы студенты получают представление о предметной области и получают возможность вносить самостоятельный вклад в работу группы.
	\item Улучшение разрабатываемой CASE-системы, стимуляция работы над важными участками функциональности, не являющимися фокусом основной разработки.
	\item Апробация новых технологий и подходов, апробация создаваемой технологии.
	\item Отбор студентов на кафедру системного программирования и поиск потенциальных кадров для Ланит-Терком.
\end{itemize}

\subsubsection{Технологии}
QReal разрабатывается с использованием следующих технологий и инструментальных средств:
\begin{itemize}
	\item Платформа: Qt~\cite{qt}. Инструментарий Qt помимо прочего содержит довольно функциональную графическую библиотеку, что является весьма полезным в разработке графических редакторов для QReal, кроме того, Qt обеспечивает кроссплатформенность на уровне перекомпиляции (write once compile everywhere).
	\item Язык программирования: C++. Выбор языка программирования обусловлен, прежде всего, историческими причинами.
	\item Операционные системы, под которыми проводится разработка: Windows, Linux
	\item Компиляторы: mingw, gcc, msvc
	\item Среды разработки: Qt Creator, vim, Visual Studio
	\item Система контроля версий: Subversion.
	\item Инструмент управления проектом, багтрекер: Trac
	\item Средства коммуникации: xmpp (jabber)
\end{itemize}

\subsubsection{Организация процесса управления}
\begin{itemize}
	\item частота личных встреч: еженедельные встречи
	\item отчётность: устный доклад на встрече, либо сообщение о статусе в xmpp-конференции проекта, не реже раза в неделю
\end{itemize}

Использовался традиционный принцип построения команды, функции управления, координации и распределения задач целиком исполняли руководители проекта. Каждому студенту одним из руководителей ставилась задача, после её решения студент отчитывался на еженедельной встрече, где ему сообщались замечания, которые необходимо исправить, или ставилась следующая задача. Студенты в процессе формулировки задач и планирования практически не участвовали. Для отслеживания статуса задач штатные средства проекта не использовались, применялась таблица в Google Docs.

\subsubsection{Коммуникации}
В качестве основного средства связи в проекте использовалась xmpp-конференция. Рассылки, форумы и т.д. не использовались совсем. Студенты практически не общались в конференции между собой, кроме того, значительная часть вопросов задавалась руководителям по jabber в обход конференции. Это может быть объяснено тем, что многие студенты не были знакомы друг с другом и боялись задавать в конференции глупые по их мнению вопросы. Руководители не препятствовали такой практике. Результатом стало то, что каждый студент занимался своей узкой задачей, не чувствуя себя частью коллектива и даже зачастую не зная, чем занимаются остальные. Тем не менее, технические вопросы решались достаточно оперативно, общение велось только по делу, и в целом, по крайней мере, в краткосрочной перспективе, такая практика зарекомендовала себя хорошо. Уже после окончания студенческого проекта, в летней школе, пришлось принимать меры для поощрения общения между студентами.

\subsubsection{Ревью, модерация коммитов}
Общие ревью кода не проводились, однако коммиты модерировались. Первое время работы над кодом непосредственно QReal каждый студент либо перед коммитом присылал одному из руководителей свои изменения, либо сразу после коммита просил посмотреть. Следует заметить, что обучение студентов особенностям использования языка программирования C++ и платформы Qt проводилось на учебных заданиях, так что коммитить в репозиторий QReal студенты начинали уже достаточно подготовленными. Учебные задачи, как правило, не коммитились в репозиторий вообще, замечания высказывались и исправлялись на месте при личных встречах. Каждый коммит в репозиторий оказывался просмотрен минимум одним из руководителей, замечания высказывались либо устно, либо письмом, либо по jabber, после чего в большинстве случаев исправлялись самими студентами.

\subsubsection{Документирование}
Техническая и пользовательская документация при проведении студпроекта практически отсутствовала. Имелось несколько статей на вики, касающихся, в основном, сборки. О том, как пользоваться системой, об общей архитектуре и деталях реализации студентам приходилось спрашивать у руководителей. Предполагалось использование системы Doxygen, и в коде наличествовало небольшое количество doxygen-комментариев, но автоматическая сборка документации появилась далеко не сразу, и в студпроекте не использовалась.

\subsubsection{Метрики}
Студенческий проект весны 2010 года:
\begin{itemize}
	\item Студенты
		\begin{itemize}
			\item Студентов пришло: 14 (из них 6 - студенты 145 группы, у которой руководители проекта вели практику по программированию, 8 - студенты 1-го курса)
			\item Студентов ушло, не успев погрузиться в проект: 5
			\item Студентов ушло, приступив к работе над проектом: 3
			\item Студентов, закончивших проект: 6 (из них студентов 145 группы: 3, 1-го курса - 5)
		\end{itemize}
	\item Строки кода
		\begin{itemize}
			\item Всего:
			\item В репозитории проекта:
		\end{itemize}
	\item Реализованная функциональность:
		\begin{itemize}
			\item Всего: решено N независимых задач 
			\item Из них полезных для проекта: M
		\end{itemize}
	\item Потраченное руководителями время: порядка N часов в неделю, M часов всего, из них на личные встречи: I, на удалённое общение: J, на ревью: K.
\end{itemize}




\subsection{Embox}

\subsubsection{Краткое описание проекта}

\subsubsection{Задачи}

\subsubsection{Технологии}

\subsubsection{Организация процесса управления}

\subsubsection{Коммуникации}

\subsubsection{Ревью, модерация коммитов}

\subsubsection{Документирование}

\subsubsection{Метрики}
\begin{itemize}
	\item Студенты
		\begin{itemize}
			\item Студентов пришло: 
			\item Студентов ушло, не успев погрузиться в проект:
			\item Студентов ушло, приступив к работе над проектом:
			\item Студентов, закончивших проект:
		\end{itemize}
	\item Строки кода
		\begin{itemize}
			\item Всего:
			\item В репозитории проекта:
		\end{itemize}
	\item Реализованная функциональность:
		\begin{itemize}
			\item Всего: 
			\item Из них полезных для проекта: 
		\end{itemize}
	\item Потраченное руководителями время: порядка N часов в неделю, M часов всего, из них на личные встречи: I, на удалённое общение: J, на ревью: K.
\end{itemize}


\subsection{Прочее}

\subsubsection{Краткое описание проекта}

\subsubsection{Задачи}

\subsubsection{Технологии}

\subsubsection{Организация процесса управления}

\subsubsection{Коммуникации}

\subsubsection{Ревью, модерация коммитов}

\subsubsection{Документирование}

\subsubsection{Метрики}
\begin{itemize}
	\item Студенты
		\begin{itemize}
			\item Студентов пришло: 
			\item Студентов ушло, не успев погрузиться в проект:
			\item Студентов ушло, приступив к работе над проектом:
			\item Студентов, закончивших проект:
		\end{itemize}
	\item Строки кода
		\begin{itemize}
			\item Всего:
			\item В репозитории проекта:
		\end{itemize}
	\item Реализованная функциональность:
		\begin{itemize}
			\item Всего: 
			\item Из них полезных для проекта: 
		\end{itemize}
	\item Потраченное руководителями время: порядка N часов в неделю, M часов всего, из них на личные встречи: I, на удалённое общение: J, на ревью: K.
\end{itemize}

\section{Анализ и выводы}

\section*{Заключение}

\pagebreak

\begin{thebibliography}{9000}
  \bibitem{qreal} А.Н. Терехов, Т.А. Брыксин, Ю.В. Литвинов и др., Архитектура среды визуального моделирования QReal. // Системное 
программирование. Вып. 4: Сб. статей / Под ред. А.Н.Терехова, Д.Ю.Булычева. --- СПб.: 2009, с. 171-196
  \bibitem{gagarsky} Р.К. Гагарский. Программа подготовки специалистов в It-компании. // Системное 
программирование. Вып. 3: Сб. статей / Под ред. А.Н.Терехова, Д.Ю.Булычева. --- СПб.: 2008, с. 141-156
  \bibitem{qt} Qt, URL: http://qt.nokia.com/ (дата обращения: 25.08.2010)
\end{thebibliography}
  

\end{document}
