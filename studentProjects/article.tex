\documentclass[a5paper]{article}
\usepackage[a5paper, top=17mm, bottom=17mm, left=17mm, right=17mm]{geometry}
\usepackage[utf8]{inputenc}
\usepackage[T2A,T1]{fontenc}
\usepackage[colorlinks,filecolor=blue,citecolor=green,unicode,pdftex]{hyperref}
\usepackage{cmap}
\usepackage[english,russian]{babel}
\usepackage{amsmath}
\usepackage{amssymb,amsfonts,textcomp}
\usepackage{color}
\usepackage{array}
\usepackage{hhline}
\hypersetup{colorlinks=true, linkcolor=blue, citecolor=blue, filecolor=blue, urlcolor=blue, pdftitle=1, pdfauthor=, pdfsubject=, pdfkeywords=}
% \usepackage[pdftex]{graphicx}
\usepackage{graphicx}
%\usepackage{epigraph}
% Раскомментировать тем, у кого этот пакет есть. Шрифт станет заметно красивее.
\usepackage{literat}
\usepackage{indentfirst}

\sloppy
\pagestyle{plain}
%\pagestyle{empty}

\title{Об управлении студенческими проектами}

\author{Т.А. Брыксин}
\date{}
\begin{document}

\maketitle
\thispagestyle{empty}

\epigraph{покайтесь! ибо конец уже близок!}%
         {Тема xmpp-конференции проекта QReal}

\begin{quote}
\small\noindent
Студенческий проект --- это некоммерческий проект по разработке программного обеспечения, в котором студенты могут под руководством опытных специалистов получить опыт в условиях, максимально приближенных к реальным промышленным или исследовательским проектам. В данной статье рассматриваются особенности организации и управления студенческими проектами с точки зрения непосредственных руководителей, приводится описание и статистика по нескольким таким проектам, сравниваются  используемые в различных проектах практики, делаются выводы об эффективности применения этих практик в зависимости от целей проекта.
\end{quote}

\section*{Введение}
Программная инженерия --- сравнительно молодая динамично развивающаяся область, поэтому к моменту завершения студентом обучения довольно большая часть полученных практических знаний успевает потерять актуальность. Кроме того, довольно многие знания и навыки, необходимые специалистам, не могут быть получены студентами из университетских курсов. Проводимые практические занятия, курсовые и дипломные работы не дают студенту достаточно практики реального  промышленного программирования, чтобы по завершении обучения быть полноценным специалистом в области программной инженерии. Одной из активностей, направленной на повышение уровня подготавливаемых специалистов, проводящейся кафедрой системного программирования СПбГУ совместно с ЗАО "Ланит-Терком", является практика проведения студенческий проектов и летних школ.

Студенческий проект представляет собой некоммерческий проект (как правило, с открытым исходным кодом), в котором группы студентов (порядка 4 - 12 человек) решают какую-либо реальную задачу. Студентами руководят, как правило, два специалиста компании, длительность студенческого проекта --- чуть менее семестра. Рабочие места студентам не предоставляются, так что большая часть работы выполняется студентами дома. Как правило, проводятся регулярные встречи участников проекта.

Компания рассматривает студенческие проекты в основном как средство отбора и подготовки кадров (подробнее об экономических аспектах подобной организации поиска кадров см. в [наверняка существующей тереховской публикации по этому вопросу, по крайней мере, доклад на сифуде Он делал]). Кафедра использует студенческие проекты также как средство отбора талантливых студентов, и для повышения уровня подготовки студентов уже отобранных. Мотивация студентов и руководителей проектов бывает различной, но в любом случае успешно прошедшие студпроект студенты могут получить рекомендацию для поступления на кафедру и на работу.





\section{Критерии сравнения проектов}

\section{Описания проектов}
\subsection{QReal}
\subsection{Embox}
\subsection{Прочее}

\section{Анализ и выводы}

\section*{Заключение}

\pagebreak

\begin{thebibliography}{9000}
  \bibitem{qreal} А.Н. Терехов, Т.А. Брыксин, Ю.В. Литвинов и др., Архитектура среды визуального моделирования QReal. // Системное 
программирование. Вып. 4: Сб. статей / Под ред. А.Н.Терехова, Д.Ю.Булычева. --- СПб.: 2009, с. 171-196
  \bibitem{gagarsky} Р.К. Гагарский. Программа подготовки специалистов в It-компании. // Системное 
программирование. Вып. 3: Сб. статей / Под ред. А.Н.Терехова, Д.Ю.Булычева. --- СПб.: 2008, с. 141-156
\end{thebibliography}
  

\end{document}
