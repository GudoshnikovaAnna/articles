\documentclass[a5paper]{article}
\usepackage[a5paper, top=17mm, bottom=17mm, left=17mm, right=17mm]{geometry}
\usepackage[utf8]{inputenc}
\usepackage[T2A,T1]{fontenc}
\usepackage[colorlinks,filecolor=blue,citecolor=green,unicode,pdftex]{hyperref}
\usepackage{cmap}
\usepackage[english,russian]{babel}
\usepackage{amsmath}
\usepackage{amssymb,amsfonts,textcomp}
\usepackage{color}
\usepackage{array}
\usepackage{hhline}
\hypersetup{colorlinks=true, linkcolor=blue, citecolor=blue, filecolor=blue, urlcolor=blue, pdftitle=1, pdfauthor=, pdfsubject=, pdfkeywords=}
% \usepackage[pdftex]{graphicx}
\usepackage{graphicx}
%\usepackage{epigraph}
% Раскомментировать тем, у кого этот пакет есть. Шрифт станет заметно красивее.
\usepackage{literat}
\usepackage{indentfirst}

\sloppy
\pagestyle{plain}
%\pagestyle{empty}

\title{Об управлении студенческими проектами}

\author{Т.А. Брыксин}
\date{}
\begin{document}

\maketitle
\thispagestyle{empty}

\epigraph{покайтесь! ибо конец уже близок!}%
         {Тема xmpp-конференции проекта QReal}

\begin{quote}
\small\noindent
Тут будет аннотация. Возможно, уже завтра.
\end{quote}

\section*{Введение}

\section{Критерии сравнения проектов}

\section{Описания проектов}
\subsection{QReal}
\subsection{Embox}
\subsection{Прочее}

\section{Анализ и выводы}

\section*{Заключение}

\pagebreak

\begin{thebibliography}{9000}
  \bibitem{qreal} А.Н. Терехов, Т.А. Брыксин, Ю.В. Литвинов и др., Архитектура среды визуального моделирования QReal. // Системное 
программирование. Вып. 4: Сб. статей / Под ред. А.Н.Терехова, Д.Ю.Булычева. --- СПб.: 2009, с. 171-196
  \bibitem{gagarsky} Р.К. Гагарский. Программа подготовки специалистов в It-компании. // Системное 
программирование. Вып. 3: Сб. статей / Под ред. А.Н.Терехова, Д.Ю.Булычева. --- СПб.: 2008, с. 141-156
\end{thebibliography}
  

\end{document}