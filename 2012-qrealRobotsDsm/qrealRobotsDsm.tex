% Статья про разработку QReal:Robots и применение DSM-подхода, для сборника кафедры

\documentclass[a4paper]{article}
\usepackage[a4paper, top=17mm, bottom=17mm, left=17mm, right=17mm]{geometry}
\usepackage[utf8]{inputenc}
\usepackage[T2A,T1]{fontenc}
\usepackage[colorlinks,filecolor=blue,citecolor=green,unicode,pdftex]{hyperref}
\usepackage{cmap}
\usepackage[english,russian]{babel}
\usepackage{amsmath}
\usepackage{amssymb,amsfonts,textcomp}
\usepackage{color}
\usepackage{array}
\usepackage{hhline}
\hypersetup{colorlinks=true, linkcolor=blue, citecolor=blue, filecolor=blue, urlcolor=blue, pdftitle=1, pdfauthor=, pdfsubject=, pdfkeywords=}
\usepackage{graphicx}
\usepackage{indentfirst}
%\usepackage{wrapfig}

\sloppy
\pagestyle{plain}

\title{Применение DSM-платформы QREAL при разработке среды программирования роботов QReal:Robots}

\author{Ю.В.Литвинов \\ ст. преп. кафедры системного программирования СПбГУ, \\ инженер-программист ЗАО ``Ланит-Терком'' \\ yurii.litvinov@gmail.com}
\date{}
\begin{document}

\maketitle
\thispagestyle{empty}

\renewcommand{\thefootnote}{}
\footnote{\small{\copyright~Ю.В.Литвинов,~2012.}}
\renewcommand{\thefootnote}{\arabic{footnote}}
\setcounter{footnote}{0}

\begin{quote}
\small\noindent
Абстракт
\end{quote}

\section*{Введение}
Сейчас в школах для преподавания информатики активно внедряются робототехнические конструкторы. Детям проще создавать свои первые программы, если они видят, как физический, осязаемый объект исполняет описанные в программе команды. Самым популярным на данный момент робототехническим набором является конструктор Lego Mindstorms NXT, он позволяет из блока управления, моторов и нескольких видов датчиков (датчики касания, расстояния, света и другие) собирать несложные устройства, которые могут исполнять команды с компьютера по интерфейсу Bluetooth или исполнять программу, загруженную на блок управления.

Для программирования этого конструктора существует несколько как текстовых, так и визуальных сред программирования. На первых этапах обучения обычно используют визуальные среды, такие как NXT-G или Robolab, однако же все популярные среды обладают рядом недостатков, делающих их использование затруднительным, таких как отсутствие полного перевода на русский язык, высокая стоимость, неудобный пользовательский интерфейс, недостаточная функциональность. Таким образом, существует потребность в разработке системы визуального программирования для Lego Mindstorms NXT, специально предназначенной для применения в школах.

Это делает область визуального программирования роботов весьма привлекательной для апробации средств предметно-ориентированного визуального программирования. На кафедре системного программирования Санкт-Петербургского Государственного Университета уже несколько десятилетий занимаются исследованиями в области визуальных языков, в частности, с 2007 года существует и активно развивается проект QReal~\ref{qReal}. В рамках этого проекта исследуются средства создания визуальных предметно-ориентированных языков, называемые также domain-specific modelling-, или DSM-средства. Поскольку визуальный язык программирования роботов является хорошим примером предметно-ориентированного языка, было решено попробовать применить имевшиеся результаты для разработки системы программирования роботов. 

В статье описано, с какими трудностями пришлось столкнуться в ходе разработки, чем помог DSM-инструментарий, чем он не смог помочь, и делаются выводы о том, какими функциями должна обладать система создания визуальных сред программирования, чтобы быть более полезной для решения реальных задач. Изложенные в статье результаты могут быть интересны тем, что разрабатываемую с помощью DSM-инструментария CASE-систему удалось довести до состояния, в котором её смогли успешно использовать люди, занимающиеся робототехникой и далёкие от программирования, в том числе и ученики пятых-шестых классов. Таким образом, по результатам этой апробации DSM-подход доказал свою применимость не только как средство создания специализированных инструментов для узкого круга пользователей, но и для разработки сред, предназначенных для массового потребителя, и используемых в образовательных и развлекательных целях. Разработанная нами среда QReal:Robots была представлена на ряде конференций и семинаров школьных учителей, роботы, запрограммированные с её помощью студентами, выступали на городских и всероссийских робототехнических фестивалях и соревнованиях. По отзывам конечных пользователей --- учеников школ --- среда QReal:Robots удобее, чем используемые ими средства программирования, и обладает более дружественным пользовательским интерфейсом.

\section{Мотивация}
Идея использовать роботов при начальном обучении информатике родилась неслучайно. Проблема, на которую указывал ещё Ф.~Брукс в своей известной книге ``Мифический человеко-месяц''~\ref{mythicalManMonth}, заключается в том, что программы нематериальны, их невозможно увидеть или потрогать. Кроме того, даже представить себе программу не так просто --- каждый человек ``видит'' программу по-разному. Людям, которые впервые пробуют программировать, приходится сразу же работать с абстрактными понятиями, и судить о правильности своих программ они могут только по внешним проявлениям их работы --- какой ответ программа выведет на экран. При этом может быть совсем не очевидно, как программа работает, что делать, если выводимый ей ответ неправильный, что нужно делать, чтобы получить правильный ответ, к тому же часто случается так, что программа работает неправильно, но правильный ответ всё-таки выводит. Всё это делает изучение информатики в школе весьма сложным.

И отечественные, и зарубежные методисты давно осознают эту проблему, поэтому традиционно начальное обучение информатике проводится с использованием концепции исполнителя: некоторого, зачастую воображаемого, устройства, способного выполнять простые команды в некотором простом окружении. Один из самых известных исполнителей, применяемых в школах --- ``черепашка'' LOGO~\ref{logo}, разработанная американским программистом, психологом и педагогом Сеймуром Пейпертом в 1967 году. Исполнитель ``черепашка'' может перемещаться по экрану, оставляя за собой след, которым вычерчиваются различные фигуры. Черепашка подчиняется командам простого интерпретируемого языка, позволяющего описывать её перемещения и повороты. Таким образом, процесс исполнения программы визуализируется движением исполнителя по экрану, и если программа работает неправильно, это будет сразу видно. 

В Советском Союзе преподавание информатики как школьного предмета началось во многом благодаря усилиям академика А.П.~Ершова и его коллектива, в который входили Г.А.~Звенигородский и Н.А.~Юнерман. Ими была разработана отечественная учебная система ``Робик''~\ref{robik}, основанная в основном на тех же принципах, что и LOGO. Ими же были разработаны методики и программы преподавания информатики в школах, где понятие ``исполнитель'' занимало ключевую позицию.

Однако исполнитель, перемещающийся по экрану, всё же недостаточно нагляден. Сеймур Пейперт в своих экспериментах использовал механическую черепашку~\ref{logoTurtle} --- реальный, осязаемый объект, исполняющий программу, гораздо понятнее для школьников, чем черепашка, движущаяся по экрану. Современные технологии позволяют создавать недорогие механические устройства, управляемые загружаемой в них программой, либо непосредственно с компьютера, поэтому идея использования материальных исполнителей в школьной информатике получила второе рождение. Самым популярным на данный момент исполнителем является кибернетический конструктор Lego Mindstorms NXT. Для преподавания информатики с использованием этого конструктора создаются методические пособия (например,~\ref{filippov}) и образовательные программы. 

Робототехнический конструктор довольно сложно программировать: поскольку из набора деталей могут быть собраны самые разные конструкции, программировать приходится в терминах оборотов моторов, подключённых к определённым портам управляющего блока, а не в терминах движения и поворотов. Это, безусловно, делает процесс обучения более творческим, поскольку школьники могут собрать своего собственного исполнителя, но и более сложным с точки зрения написания для этого исполнителя программ. Проблема сложности программирования преодолевается использованием наглядных визуальных языков и удобных графических редакторов для составления программ из блоков, представляющих элементарные команды, такие как ``включить мотор'', ``гудок'' и т.д. Таким образом, начинающие работают с графическими языками программирования, а более опытные школьники постепено переходят на текстовые C-образные языки. В комплекте с конструктором поставляется графическая среда программирования NXT-G, поэтому визуальные языки среди использующих Mindstorms NXT весьма популярны.

В проекте QReal ведутся исследования в области визуальных языков, в частности, в области предметно-ориентированного визуального программирования. Предметно-ориентированный подход к созданию программного обеспечения основан на том, что иногда задачу проще решать, создав для её решения специализированный язык и решая задачу на нём, чем использовать язык общего назначения. При этом в исследовательской группе, работающей над проектом QReal, полагают, что визуальные языки гораздо более наглядны, чем текстовые, поэтому больше подходят как средства предметно-ориентированного программирования. Разумеется, создавать визуальный язык и средства поддержки для него (такие как редактор диаграмм, генераторы кода) с нуля было бы слишком трудозатратно, что свело бы на нет все преимущества от узкой специализированности языка, поэтому создаются средства, позволяющие автоматизировать часть этого процесса. Если обычные среды для визуального моделирования и визуального программирования называют CASE-средствами, то средства для создания таких сред называют обычно metaCASE-средствами, QReal --- пример metaCASE-системы.

Программирование роботов является интересной предметной областью для апробации metaCASE-системы. С одной стороны, использование визуальных языков программирования довольно широко распространено среди людей, занимающихся робототехникой, поэтому можно рассчитывать на содержательное сравнение с существующими решениями и содержательные отзывы пользователей, уже знакомых с различными визуальными языками. С другой стороны, программирование роботов --- хороший пример предметной области, достаточно узкой, чтобы можно было получить заметное преимущество от создания для неё специализированного языка, и вместе с тем достаточно содержательной, чтобы такой специализированный язык был нетривиальным, и мог бы послужить примером для исследования вопросов, применимых в других содержательных случаях. Типичные программы для роботов состоят из элементарных команд роботу, таких как "включить моторы", "ожидать такое-то показание сенсора" и т.д., и управляющих конструкций, таких как условные операторы и циклы. На языке общего назначения такие команды были бы вызовами API операционной системы или библиотек робота, и требовали бы для себя различных вспомогательных конструкций, таких как операторы включения и объявления переменных, а на специализированном языке каждая такая команда представляется одним блоком, для использования которого достаточно просто разместить его на диаграмме. Это существенно снижает требования к знаниям программиста и снижает вероятность ошибки в программе --- например, невозможно опечататься при указании имени вызываемой функции. Вместе с тем, многие языковые конструкции, типичные для императивного программирования, такие как ветвления и циклы, будут присутствовать и в этом языке, так что если удастся подобрать удобное графическое представление для программ для роботов, можно надеяться обобщить полученный результат на другие задачи, хорошо выражаемые в императивных терминах.

\section{Средства визуального программирования роботов}
Рассмотрим существующие средства визуального программирования роботов. Поскольку в контексте данной работы нас интересует использование роботов в школьной информатике, основной акцент в обзоре будет сделан на свойства продуктов, важные в этой сфере их применения:
\begin{itemize}
  \item они должны позволять писать довольно сложные программы, включающие в себя нетривиальные математические выражения, циклы, ветвления, переменные, параллельные задачи --- применение таких средств должно дать возможность иллюстрировать содержательный материал из информатики и кибернетики, например, понятие регуляторов;
  \item они должны быть просты и удобны в работе, потому как неудобный пользовательский интерфейс создавал бы дополнительную когнитивную нагрузку для школьников и усложнял бы восприятие и без того сложного материала;
  \item они должны иметь встроенные средства отладки, чтобы школьники могли следить за ходом выполнения своей программы и её состоянием, имели бы инструмент для эффективного поиска ошибок;
  \item желательна возможность перехода от графической формы программы к текстовой, чтобы школьники старших классов, серьёзно занимающиеся программированием, имели возможность смотреть, как их программа выглядит на более приближенном к индустриальному программированию текстовом языке, возможно, вносить в текстовую программу правки, в той же среде, в которой они привыкли работать;
  \item необходима русификация среды разработки, поскольку школьники зачастую ещё не владеют иностранными языками, а необходимость работать со словарём существенно усложняет восприятие материала;
  \item безусловно, важное свойство продукта --- его цена, каким бы хорошим ни был тот или иной продукт, если он стоит дорого, не все школы могут себе его позволить;
  \item желательно, чтобы продукт продолжал развиваться и адаптироваться к новым операционным системам и аппаратному обеспечению.
\end{itemize}
Основные среды программирования, распространённые на данный момент, рассмотрены ниже с точки зрения перечисленных критериев.

\subsection{Среда NXT-G}
Среда NXT-G --- единственное средство програмимрования, которое поставляется в комплекте с конструктором Lego Mindstorms NXT. Среда базируется на LabView, среде визуального программирования от компании National Instruments. В LabView в качестве языка программирования используется визуальный язык G (из-за чего NXT-G и получила своё название). Язык G моделирует процесс вычислений, ориентированный на поток данных, при котором явно задаётся не последовательность выполнения операторов, а связи между блоками по данным. Блок языка может предоставлять некоторые выходные данные, которые могут служить входными данными для другого блока. Блоки начинают исполняться, когда имеют данные на всех входах. Если сразу несколько блоков имеют данные на всех входах, они исполняются параллельно. Такой подход довольно сильно отличается от подхода, принятого в визуальных языках на основе блок-схем, и вообще от подхода, принятого в императивном программировании, однако же довольно распространён среди инженеров и учёных, например, такому же принципу следует другая известная визуальная среда програмирования научных вычислений и моделирования Matlab/Simulink.

Основная проблема этой среды состоит в довольно слабой поддержке математических выражений. Математические формулы здесь, как и вся программа, строятся из блоков. Есть блоки арифметических операций, блоки чтения и записи значения в переменную, блок, считывающий значение константы, блоки, считывающие показания с сенсоров. Таким образом, даже чтобы запрограммировать несложную формулу, требуется изображать блоками фактически дерево разбора выражения, которое эту формулу задаёт. Для иллюстрации серьёзности этой проблемы достаточно сказать, что программа, представляющая пропорционально-дифференциальный регулятор для движения робота вдоль линии или вокруг препятствия, на языке C занимает порядка десятка строк, тогда как на NXT-G не помещается на одном экране и весьма сложна для понимания. Таким образом, первому из предложенных критериев --- пригодности для иллюстрации содержательного материала информатики и кибернетики, NXT-G не соответствует. В основном поэтому NXT-G и не получил широкого распространения в школах.

Что касается простоты и удобства в работе, NXT-G специально создавался для начинающих и поэтому довольно эргономичен. Мо мнению некоторых пользователей слишком эргономичен, поскольку не даёт произвольно размещать блоки на диаграмме, автоматически (и не всегда удачно) прокладывает соединительные линии между блоками, и т.д. Для применения NXT-G в школьных классах оказалась важна ещё такая его особенность: большая часть свойств элементов не отображается на диаграмме, а доступна только через редактор свойств, что делает невозможным показ всей диаграммы на проекторе. Никаких средств отладки NXT-G не имеет, текстовая форма программы не порождается, русификация существует, но неофициальная. К плюсам продукта следует отнести то, что он распространяется вместе с конструктором, и доступен для скачивания с сайта производителя бесплатно. Кроме того, продукт до сих пор развивается и обновляется. Средствами LabView возможно добавление сторонних блоков, кроме того, сам NXT-G позволяет выделить набор блоков в подпрограмму и использовать её как новый блок.

\subsection{Среда Robolab}
Среда Robolab~\ref{robolab} так же, как и NXT-G, базируется на среде LabView. Эта среда специально создавалась для школьного образования, и самого начала своего развития учитывала пожелания школьных учителей и специфику преподавания в школах. Пример специфичного для школ решения, реализованного в Robolab --- наличие нескольких уровней возможностей среды. На самом простом уровне доступны только некоторые возможности визуального языка, и программа строится заполнением пустых мест в шаблоне выбором блоков из всплывающего меню. Это позволяет писать только самые простые программы, имеющие стандартную структуру: команды управления моторами, за которыми следует блок, ожидающий наступления какого-либо события. Причём, этот уровень имеет четыре подуровня, и на первых трёх подуровнях программа может состоять только из одного такого "шага". Сделано это для того, чтобы дать возможность детям в начальной школе или даже детском саду пользоваться программой --- в столь раннем возрасте они вполне могут не уметь читать. На втором уровне (который тоже состоит из нескольких подуровней) пользователи могут рисовать уже настоящие диаграммы, размещая произвольным образом блоки из палитры и соединяя их линиями, определяющими поток управления. Разница между подуровнями заключается в количестве доступных в палитре блоков, первые подуровни имеют меньше блоков с меньшим количеством параметров. Разбиение на уровни и подуровни организовано так, чтобы дети могли осваивать среду программирования практически без помощи учителя, руководствуясь лишь интуицией. Отзывы учителей, приведённые в~\ref{robolab}, показывают, что этой цели удалось достигнуть.

Математические выражения Robolab поддерживает гораздо лучше, чем NXT-G, позволяя писать произвольные выражения в текстовом виде. В Robolab имеется возможность в формулах использовать тригонометрические функции, обращаться напрямую к значениям показаний сенсоров. Циклы в Robolab реализованы довольно необычно --- есть блок "метка" и блок "переход к метке", передача управления никак больше не визуализируется. Имеются условные операторы, возможность порождать параллельные процессы, блоки для управления этими процессами, работа с подпрограммами. На Robolab можно просто и довольно удобно реализовать даже довольно сложные программы, и Robolab вполне подходит для иллюстрации материала из кибернетики до младших курсов ВУЗов.

По другим критериям Robolab показывает несколько худшие характеристики. Программа была создана в конце 90-х годов, с тех пор интерфейс значительно не менялся, поэтому сейчас выглядит несовременно, и довольно неудобен. Специализированных средств отладки в Robolab нет, хотя есть возможность снимать показания с робота и отображать на экране компьютера данные. Текстовой формы программы Robolab порождать не может. Русификация присутствует, но только частичная, некоторые элементы управления не переведены. Robolab небесплатен, одна лицензия по стоимости сравнима с робототехническим набором, что для школ довольно дорого. Развитие Robolab идёт в основном путём добавления новых блоков, сама среда давно не изменялась.

Несмотря на указанные недостатки, Robolab на данный момент является основной используемой в школах средой разработки программ для роботов. По отзывам учителей, у многих имеется желание от него отказаться и заменить на что-нибудь более современное, однако пока на рынке не существует продуктов, которые могли бы составить ему серьёзную конкуренцию.

\subsection{Среда Microsoft Robotics Developer Studio}
Среда Microsoft Robotics Developer Studio~\ref{mrds} --- разработка компании Microsoft, предназначенная для программирования сложных многопоточных приложений с реактивной моделью поведения, используемых для управления робототехническими системами. Необходимость создания таких приложений есть не только в робототехнике, поэтому Robotics Developer Studio используется и для создания приложений, к робототехнике не относящихся (например, социальная сеть MySpace использует MRDS как составную часть серверного ПО~\ref{mrdsAtMySpace}). Программы в Robotics Developer Studio рисуются в виде диаграмм на визуальном языке VPL (Visual Programming Language), представляющем собой по сути визуализатор связей между отдельными параллельно исполняемыми компонентами (или веб-сервисами), из которых состоит программа. Система состоит из следующих крупных частей.

\begin{description}
  \item[Concurrency and Coordination Runtime (CCR)] --- библиотека для работы с паралельными и асинхронными потоками данных. Библиотека автоматизирует синхронизацию процессов, основываясь на потоках данных, так что прикладному программисту не надо задумываться об использовании семафоров, мониторов и прочих примитивов синхронизации. Библиотека позволяет прозрачно организовывать распределённые и параллельные вычисления, исполняя задачи на разных вычислительных устройствах. Это весьма полезно при программировании роботов, поскольку программы для роботов по природе реактивны и требуют обработки потоков данных одновременно с нескольких сенсоров, причём часть вычислений может быть сделана прямо на роботе, часть --- на компьютере вне него.
  \item[Decentralized Software Services (DSS)] --- среда времени выполнения, обеспечивающая представление компонентов программы в виде веб-сервисов и упрощающая организацию взаимодействия между ними. Взаимодействие между веб-сервисами ведётся по специальному протоколу Decentralized Software Services Protocol (DSSP). Инструменты, входящие в DSS, позволяют довольно легко конфигурировать веб-сервисы и связи между ними. DSS позволяет создавать распределённые приложения, которым не важно, на каком вычислительном устройстве выполняется тот или иной компонент, на одном из компьютеров робота или на компьютере вовне, лишь бы они были связаны единой сетью.
  \item[Visual Programming Language (VPL)] --- визуальный язык и редактор для него, используемый для конфигурирования сервисов. Сервисы можно перетащить на диаграмму, связать их входы и выходы, настроить их атрибуты. Получающаяся на таком языке диаграмма сильно напоминает диаграммы LabView, отображая зависимость между компонентами по данным.
  \item[Visual Simulation Environment (VSE)] --- трёхмерная среда симуляции поведения робота в виртуальном мире. Обладает довольно богатыми возможностями по симуляции физике, и богатыми средствами отображения трёхмерной графики, что позволяет строить сложные и красиво выглядящие модели мира, с которым робот может взаимодействовать. В поставку среды включено несколько моделей окружения, в том числе модель квартиры, в которой работает так называемая "стандартная модель" робота, трёхколёсная платформа с установленным на ней ноутбуком, сенсором Microsoft Kinect\textregistered, инфракрасными датчиками расстояния и сонаром.
\end{description}

В школьном образовании, однако же, Microsoft Robotics Developer Studio используется очень редко. Причин этому несколько, главная из которых --- среда рассчитана в основном на симуляцию, и не может эффективно взаимодействовать с реальным роботом. Для LEGO Mindstorms NXT есть возможность управления по каналу Bluetooth, но залить программу на робот возможности нет --- на роботе нет возможности запустить .NET-машину, и нет ресурсов, необходимых для работы распределённых веб-сервисов. Реальные роботы, управляемые MRDS, обычно гораздо сложнее и дороже того, что можно использовать в школах (стандартная платформа, например, имеет в своём составе ноутбук, который один, скорее всего, дороже всего набора Mindstorms, хотя позиционируется как бюджетный робот, доступный каждому). Управления по Bluetooth недостаточно для решения задач, требующих малого времени реакции робота, из-за больших задержек посылки-приёма Bluetooth-пакетов, что делает MRDS неприменимой для большой области решаемых в школе задач. Симуляции же тоже оказывается недостаточно, потому что даже с хорошим физическим движком MRDS создаёт модель некоторого идеального мира, в котором большого количества проблем, решаемых алгоритмами кибернетики, просто не возникает. Даже простая задача, решаемая на реальном роботе, может оказаться нагляднее и полезнее школьникам, чем сложная программа, исполняемая на модели в симуляторе. А поскольку возможности быстро перейти от симулятора к реальному роботу нет, MRDS скорее занимает нишу "черепашки" Logo, применяясь в основном в виде исполнителя на экране.

Вторая важная причина очень узкого распространения MRDS в школах --- модель вычислений, в ней используемая. Представление программы в виде набора взаимосвязанных распределённых веб-сервисов может быть удобным для опытных программистов, но начинающим тяжело понять принципы, лежащие в основе такой модели программирования. Сложные механизмы взаимодействия веб-сервисов во многом спрятаны с помощью визуального языка VPL, но всё же требуется некоторое понимание происходящих в системе процессов для того, чтобы рисовать содержательные диаграммы. К тому же, блоки имеют довольно большое количество параметров, при задании которых используются различные сложные алгоритмические понятия, незнакомые начинающим программистам. В целом можно сказать, что MRDS больше подходит для студентов или профессиональных программистов, чем для школьников.

\section{Предметно-ориентированное моделирование и система QReal}

\section{Среда QReal:Robots}

\section{Применение QReal в разработке QReal:Robots}

\section*{Заключение}
В результате применения описанного в статье подхода удалось разработать полноценную среду программирования роботов, которую оказалось возможным предложить школьникам и учителям в качестве замены используемых ныне в школах средств. QReal:Robots была представлена на "Открытых состязаниях Санкт-Петербурга по робототехнике" и на робототехническом фестивале "Робофест 2012" в Москве. В качестве доказательства применимости среды QReal:Robots к реальным задачам, решаемым школьниками, команда студентов приняла участие в соревнованиях в движении робота по линии с программой, реализованной целиком на QReal:Robots. Задача движения по линии состоит в том, что руководствуясь показанием датчиков освещённости робот должен проехать по нарисованной на полу кривой замкнутой чёрной линии за возможно меньшее время. Несмотря на то, что для студентов участие в соревнованиях стало практически первым опытом решения задач робототехники, им удалось показать довольно неплохие результаты, заняв места в середине таблицы, несмотря на то, что довольно многие участники использовали специально созданные для этой задачи роботы. QReal:Robots представлялся также в виде стендовых докладов, где вызвал большую заинтересованность у потенциальных пользователей. Было проведено анкетирование удобства пользовательского интерфейса, которое показало, что продукт достаточно хорош, чтобы вызывать у пользователей симпатию и желание им пользоваться.

Среду QReal:Robots вряд ли удалось бы разработать в столь короткие сроки вручную. Использование metaCASE-системы QReal позволило существенно упростить создание визуального языка и редактора диаграмм для него. Прототип языка был готов за несколько часов, так что уже после недели разработки первый функциональный прототип среды программирования, включающий в себя редактор диаграмм и интерпретатор, управляющий роботом по Bluetooth, был представлен специалистам по кибернетике. При создании языка наиболее затратным по времени оказался поиск подходящих иконок для блоков диаграммы. Создание метамодели языка с помощью метаредактора оказалось довольно быстрым, первая версия языка содержала порядка десятка блоков, находящихся друг с другом в несложных отношениях, и с тех пор язык лишь незначительно вырос, вся его метамодель в графическом виде помещается на один экран. Поскольку редактор диаграмм языка по метамодели генерируется автоматически, была возможность не только сэкономить время на разработке редактора, но и экспериментировать с языком, меняя блоки, их свойства и внешний вид. Кроме того, простота визуального представления синтаксиса языка дала возможность легко расширять и изменять его в дальнейшем, причём не только исходным авторам языка. Впоследствии разработка QReal:Robots была во многом передана студентам, которые без особых сложностей редактировали и расширяли язык по мере необходимости. Рассматривалась даже возможность позволить самим пользвателям менять язык, например, дать возможность учителям информатики настраивать язык под конкретное занятие, но такая возможность до сих пор не была реализована в силу серьёзной недоработанности средств метамоделирования в QReal.

Если синтаксис языка оказался хорошо формализуемым и процесс построения редактора языка по описанию синтаксиса удалось автоматизировать, то с семантикой ситуация оказалась хуже. Система QReal на момент разработки QReal:Robots не имела никаких средств описания семантики языка, так что вся инструментальная поддержка, включая интерпретатор диаграмм и генератор кода для заливки на робот, реализовывалась вручную, программированием на C++. Эта задача трудоёмка и сама по себе --- требовалось изучить систему прямых команд робота для управления им с компьютера, изучить API операционной системы робота для генерации кода для неё, наладить работу с Bluetooth, USB, кросскомпиляцию, прошивку робота, загрузку программы на робот, и прочие проблемы, не автоматизируемые в принципе. Однако же, некоторый код получился довольно шаблонным --- каждому блоку соответствовал некоторый класс на С++, реализовывавший его функциональность при интерпретации, и по крайней мере шаблоны для таких классов можно было бы генерировать автоматически по описанию языка. Интерпретатор в целом также мало связан конкретно с языком программирования роботов, так что, возможно, часть его можно было бы генерировать по описанию семантики языка, представленной в каком-то виде, а часть вынести в библиотеку и использовать в сгенерированном коде. Предполагается, что использование интерпретируемых языков программирования для описания части поведения интерпретатора позволило бы ускорить разработку, но эта идея ещё не была проверена.

Генератор кода представляется наиболее интересной целью для автоматизации, поскольку содержит много похожего от блока к блоку кода. Как и интерпретатор, генератор содержит неспецифичные для языка программирования роботов части, которые были бы применимы для всех визуальных языков с семантикой, похожей на семантику диаграмм активностей UML или блок-схем. Над автоматизацией создания генератора в QReal в данный момент ведётся работа, был создан язык описания правил генерации, берущий на себя многие типичные задачи, такие как обход графа модели и вывод текста. Тем не менее, такой язык не позволяет выполнять сложные алгоритмические действия, например, goto elimination, или поиск переменных в математических выражениях для дальнейшей автоматической генерации блока объявления переменных.

Таким образом, результатом эксперимента по разработке системы визуального программирования с помощью metaCASE-системы стало то, что с помощью такого подхода можно создать систему, которая была бы не хуже разработанных вручную, и при этом время на разработку визуального редактора с помощью метамоделирования можно сократить в несколько десятков раз по сравнению с разработкой редактора вручную. Однако выигрыш при разработке и доведении до внедрения системы в целом оказался не таким значительным, как при разработке только редактора, потому как многие задачи принципиально неавтоматизируемы. В процессе эксперимента были выявлены типичные при разработке подобного рода систем задачи, которые можно в некоторой степени автоматизировать в metaCASE-системе, но их автоматизация требует дополнительных исследований. Большая часть таких задач требует изучения способов задания семантики визуальных языков. Однако даже сейчас DSM-подход показал себя полезным на практике, и представляется, что дальнейшие исследования в этом направлении смогут помочь эффективно создавать специализированные языки и среды программирования даже для гораздо более узких предметных областей и задач.

\begin{thebibliography}{9001}

  \bibitem{robots} \emph{Брыксин Т.А., Литвинов Ю.В.} Среда визуального программирования роботов QReal:Robots // Материалы международной конференции ``Информационные технологии в образовании и науке''. Самара. 2011. С. 332--334.
  
  \bibitem{logoTurtle} http://cyberneticzoo.com/?p=1711
  
  \bibitem{robolab} \emph{Portsmore, Merredith} ROBOLAB: Intuitive Robotic Programming Software to Support Life Long Learning, APPLE Learning Technology Review, Spring/Summer 1999
  
  \bibitem{mrdsAtMySpace} \emph{Michael S. Scherotter}, CCR at MySpace, http://channel9.msdn.com/Shows/Communicating/CCR-at-MySpace (дата обращения: 01.07.2012)

\end{thebibliography}

\end{document}
