% Статья про разработку QReal:Robots и применение DSM-подхода, для сборника кафедры

\documentclass[a4paper]{article}
\usepackage[a4paper, top=17mm, bottom=17mm, left=17mm, right=17mm]{geometry}
\usepackage[utf8]{inputenc}
\usepackage[T2A,T1]{fontenc}
\usepackage[colorlinks,filecolor=blue,citecolor=green,unicode,pdftex]{hyperref}
\usepackage{cmap}
\usepackage[english,russian]{babel}
\usepackage{amsmath}
\usepackage{amssymb,amsfonts,textcomp}
\usepackage{color}
\usepackage{array}
\usepackage{hhline}
\hypersetup{colorlinks=true, linkcolor=blue, citecolor=blue, filecolor=blue, urlcolor=blue, pdftitle=1, pdfauthor=, pdfsubject=, pdfkeywords=}
\usepackage{graphicx}
\usepackage{indentfirst}
%\usepackage{wrapfig}

\sloppy
\pagestyle{plain}

\title{Применение DSM-платформы QREAL при разработке среды программирования роботов QReal:Robots}

\author{Ю.В.Литвинов \\ ст. преп. кафедры системного программирования СПбГУ, \\ инженер-программист ЗАО ``Ланит-Терком'' \\ yurii.litvinov@gmail.com}
\date{}
\begin{document}

\maketitle
\thispagestyle{empty}

\renewcommand{\thefootnote}{}
\footnote{\small{\copyright~Ю.В.Литвинов,~2012.}}
\renewcommand{\thefootnote}{\arabic{footnote}}
\setcounter{footnote}{0}

\begin{quote}
\small\noindent
Абстракт
\end{quote}

\section*{Введение}
Идея использовать роботов при начальном обучении информатике родилась неслучайно. Проблема, на которую указывал ещё Ф.~Брукс в своей известной книге ``Мифический человеко-месяц''~\ref{mythicalManMonth}, заключается в том, что программы нематериальны, их невозможно увидеть или потрогать. Кроме того, даже представить себе программу не так просто --- каждый человек ``видит'' программу по-разному. Людям, которые впервые пробуют программировать, приходится сразу же работать с абстрактными понятиями, и судить о правильности своих программ они могут только по внешним проявлениям их работы --- какой ответ программа выведет на экран. При этом может быть совсем не очевидно, как программа работает, что делать, если выводимый ей ответ неправильный, что нужно делать, чтобы получить правильный ответ, к тому же часто случается так, что программа работает неправильно, но правильный ответ всё-таки выводит. Всё это делает изучение информатики в школе весьма сложным.

И отечественные, и зарубежные методисты давно осознают эту проблему, поэтому традиционно начальное обучение информатике проводится с использованием концепции исполнителя: некоторого, зачастую воображаемого, устройства, способного выполнять простые команды в некотором простом окружении. Один из самых известных исполнителей, применяемых в школах --- ``черепашка'' LOGO~\ref{logo}, разработанная американским программистом, психологом и педагогом Сеймуром Пейпертом в 1967 году. Исполнитель ``черепашка'' может перемещаться по экрану, оставляя за собой след, которым вычерчиваются различные фигуры. Черепашка подчиняется командам простого интерпретируемого языка, позволяющего описывать её перемещения и повороты. Таким образом, процесс исполнения программы визуализируется движением исполнителя по экрану, и если программа работает неправильно, это будет сразу видно. 

В Советском Союзе преподавание информатики как школьного предмета началось во многом благодаря усилиям академика А.П.~Ершова и его коллектива, в который входили Г.А.~Звенигородский и Н.А.~Юнерман. Ими была разработана отечественная учебная система ``Робик''~\ref{robik}, основанная в основном на тех же принципах, что и LOGO. Ими же были разработаны методики и программы преподавания информатики в школах, где понятие ``исполнитель'' занимало ключевую позицию.

Однако, исполнитель, перемещающийся по экрану, всё же недостаточно нагляден. Сеймур Пейперт в своих экспериментах использовал механическую черепашку~\ref{logoTurtle} --- реальный, осязаемый объект, исполняющий программу, гораздо понятнее для школьников, чем черепашка, движущаяся по экрану. Современные технологии позволяют создавать недорогие механические устройства, управляемые загружаемой в них программой, либо непосредственно с компьютера, поэтому идея использования материальных исполнителей в школьной информатике получила второе рождение. Самым популярным на данный момент исполнителем является кибернетический конструктор Lego Mindstorms NXT, активно внедряемый сейчас в российских школах. Для преподавания информатики с использованием этого конструктора создаются методические пособия (например,~\ref{filippov}) и образовательные программы. 

Робототехнический конструктор довольно сложно программировать: поскольку из набора деталей могут быть собраны самые разные конструкции, программировать приходится в терминах оборотов моторов, подключённых к определённым портам управляющего блока, а не в терминах движения и поворотов. Это, безусловно, делает процесс обучения более творческим, поскольку школьники могут собрать своего собственного исполнителя, но и более сложным с точки зрения написания для этого исполнителя программ. Проблема сложности программирования преодолевается использованием наглядных визуальных языков и удобных графических редакторов для составления программ из блоков, представляющих элементарные команды, такие как ``включить мотор'', ``гудок'' и т.д. Таким образом, начинающие работают с графическими языками программирования (которых сейчас для Mindstorms NXT существует достаточно много), а более опытные школьники постепено переходят на текстовые C-образные языки. В комплекте с конструктором поставляется графическая среда программирования NXT-G, поэтому визуальные языки среди использующих Mindstorms NXT весьма популярны.

На кафедре системного программирования Санкт-Петербургского Государственного Университета уже несколько десятилетий занимаются исследованиями в области визуальных языков, в частности, с 2007 года существует и активно развивается проект QReal~\ref{qReal}. В рамках этого проекта исследуются средства создания визуальных предметно-ориентированных языков. Предметно-ориентированный подход к созданию программного обеспечения основан на том, что иногда задачу проще решать, создав для её решения специализированный язык и решая задачу на нём, чем использовать язык общего назначения. При этом в исследовательской группе, работающей над проектом QReal, полагают, что визуальные языки гораздо более наглядны, чем текстовые, поэтому больше подходят как средства предметно-ориентированного программирования. Разумеется, создавать визуальный язык и средства поддержки для него (такие как редактор диаграмм, генераторы кода) с нуля было бы слишком трудозатратно, что свело бы на нет все преимущества от узкой специализированности языка, поэтому создаются средства, позволяющие автоматизировать часть этого процесса. Если обычные среды для визуального моделирования и визуального программирования называют CASE-средствами, то средства для создания таких сред называют обычно metaCASE-средствами, QReal --- пример metaCASE-системы.

Программирование роботов является интересной предметной областью для апробации metaCASE-системы. С одной стороны, использование визуальных языков программирования довольно широко распространено среди людей, занимающихся робототеникой, поэтому можно рассчитывать на содержательное сравнение с существующими решениями и содержательные отзывы пользователей, уже знакомых с различными визуальными языками. С другой стороны, программирование роботов --- хороший пример предметной области, достаточно узкой, чтобы можно было получить заметное преимущество от создания для неё специализированного языка, и вместе с тем достаточно содержательной, чтобы такой специализированный язык был нетривиальным, и мог бы послужить примером для исследования вопросов, применимых в других содержательных случаях. Типичные программы для роботов состоят из элементарных команд роботу, таких как "включить моторы", "ожидать такое-то показание сенсора" и т.д., и управляющих конструкций, таких как условные операторы и циклы. На языке общего назначения такие команды были бы вызовами API операционной системы или библиотек робота, и требовали бы для себя различных вспомогательных конструкций, таких как операторы включения и объявления переменных, а на специализированном языке каждая такая команда представляется одним блоком, для использования которого достаточно просто разместить его на диаграмме. Это существенно снижает требования к знаниям программиста и снижает вероятность ошибки в программе --- например, невозможно опечататься при указании имени вызываемой функции. Вместе с тем, многие языковые конструкции, типичные для императивного программирования, такие как ветвления и циклы, будут присутствовать и в этом языке, так что если удастся подобрать удобное графическое представление для программ для роботов, можно надеяться обобщить полученный результат на другие задачи, хорошо выражаемые в императивных терминах.

Было логичным попробовать применить разрабатываемую в том числе и автором данной статьи metaCASE-систему QReal к возникшей задаче программирования роботов, тем более что с предложением создать такую среду на основе QReal обратились специалисты по робототехнике. Довольно быстро (за время порядка нескольких часов) с использованием редактора визуальных языков из состава QReal был создан прототип редактора визуального языка программирования роботов, затем несколько недель ушло на разработку инструментальной поддержки для него, включающей в себя интерпретатор, управляющий роботом Lego NXT по Bluetooth-интерфейсу, отладочный эмулятор и двухмерную модель робота. В дальнейшем язык и инструментальные средства вокруг него совершенствовались и расширялись, задачам и трудностям, возникшим при этом, и роли DSM-инструментария в процессе разработки посвящена дальнейшая часть статьи.

Изложенные в данной статье результаты могут быть интересны тем, что разрабатываемую с помощью DSM-инструментария CASE-систему удалось довести до состояния, в котором её смогли успешно использовать люди, занимающиеся робототехникой и далёкие от программирования, в том числе и ученики пятых-шестых классов. Таким образом, по результатам этой апробации DSM-подход доказал свою применимость не только как средство создания специализированных инструментов для узкого круга пользователей, но и для разработки сред, предназначенных для массового потребителя, и используемых в образовательных и развлекательных целях. Среда QReal:Robots была представлена на ряде конференций и семинаров школьных учителей, роботы, запрограммированные с её помощью студентами, выступали на городских и всероссийских робототехнических фестивалях и соревнованиях. По отзывам конечных пользователей --- учеников школ --- среда QReal:Robots удобее, чем используемые ими средства программирования, и обладает более дружественным пользовательским интерфейсом.

\section{Средства визуального программирования роботов}

\section{DSM и QReal}

\section{QReal:Robots}

\section{Применение QReal в разработке QReal:Robots}

\section*{Заключение}

\begin{thebibliography}{9001}

  \bibitem{robots} \emph{Брыксин Т.А., Литвинов Ю.В.} Среда визуального программирования роботов QReal:Robots // Материалы международной конференции ``Информационные технологии в образовании и науке''. Самара. 2011. С. 332--334.
  
  \bibitem{logoTurtle} http://cyberneticzoo.com/?p=1711

\end{thebibliography}

\end{document}
